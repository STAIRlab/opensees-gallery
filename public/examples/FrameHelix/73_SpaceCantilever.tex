%
\hypertarget{sec:helical}{%
\subsection{Spatial Response of Cantilever Beam Under End Moment and End Force}\label{sec:helical}}
%
\begin{figure*}
\centerline{%
\begin{subfigure}[b]{0.48\textwidth}
\centering
\includegraphics[width=0.9\textwidth,keepaspectratio]{Figures/Figure_2a}
\end{subfigure}
\begin{subfigure}[b]{0.6\textwidth}
\centering
\includegraphics[width=0.9\textwidth,keepaspectratio]{Figures/Figure_2b}
\end{subfigure}
}
\caption{Deformed shape of the cantilever beam bent into a helical form at \(\lambda=1\) (left) alongside the relation between the tip displacement in the loading direction and the load factor $\lambda$ (right).}\label{fig:helix}
\end{figure*}
%
The cantilever beam 
from Section~\ref{sec:plane} is now subjected to a \emph{combined} point moment \(\boldsymbol{M}\) 
and a point force \(F \, \mathbf{E}_3\) at its free end \(\xi=L\).
%
This example is selected to demonstrate the ability of the proposed formulations 
to naturally accommodate applied moments in various reference frames.
It also highlights the accuracy and convergence characteristics of the formulations.
% how different rotation parameterizations are work-conjugate to moments of fundamentally distinct natures. 
Three common variations of this problem are considered
with the properties from Section~\ref{sec:circle} for all cases
\footnote{These parameters were used by \cite{ritto-corrêa2002differentiation,ibrahimbegovic1997choice}.

It is reported in \cite{ibrahimbegović1995computational} that an axial stiffness of \(EA=2GA\)
was used for simulation, but \cite{ritto-corrêa2002differentiation} observe that this may be a reporting error. The authors believe that the simulations of \cite{ibrahimbegović1995computational}
have been performed with the parameters of the present study.
}:
\[
\begin{array}{lcr}
    L  &=&    10\hphantom{..}    \\ % ,& A  &= 1 \\
    E  &=&    10^4  \\ % ,& I  &= 10^{-2} \\
    G  &=&    10^4  \\ % ,& J  &= 10^{-2} \\
\end{array}
\qquad\qquad
\begin{array}{lcr}
    A  &=& 1\hphantom{..} \\
    I  &=& 10^{-2} \\
    J  &=& 10^{-2} \\
\end{array}
\]
%
%
\subsubsection{Simple Perturbation}
Following \cite{ibrahimbegović1995computational}, the problem of plane flexure from Section~\ref{sec:circle} is now altered by introducing the point load
\(\boldsymbol{F} = 1/16 \, \mathbf{E}_3\) in addition to the moment \(\boldsymbol{M}\) so as to induce
a three-dimensional response. 
A uniform mesh of 10, 2-node elements is used, and the reference moment in Equation~(\ref{eq:fref}) for \(\lambda = 1/8\) is applied in a single load step. 
Because the deformation is no longer plane, each choice of nodal parameterization essentially equilibriates the moment in a different coordinate system. %solves a different problem.
Results are reported in Table~\ref{tab:helical-perturb01}, where the \texttt{None/None/None} and \texttt{Incr/None/Incr} 
variants match the values reported by \cite{ibrahimbegović1995computational} for the formulations by 
\cite{simo1986threedimensional} and \cite{ibrahimbegović1995computational}, respectively.
Once again, the application of external isometry or parameter transformations does not affect the convergence characteristics of the solution.
%
\input{Tables/CantileverPerturbed01}

\subsubsection{Consistent Perturbation}
The problem is simulated again, but now the moment is consistently applied with a spatial orientation.
Formulations whose final residual moment vector is conjugate to the spatial variations \(\boldsymbol{u}_{\scriptscriptstyle{\Lambda}}\) of the rotation \(\boldsymbol{\Lambda}\) do not need to be treated differently.
This includes both elements with the \texttt{None} parameter transformation and transformed elements with the Petrov-Galerkin formulation.
For the simulations with all other elements, a transformation of the nodal force is necessary, as described in \cite{ritto-corrêa2002differentiation,ritto-corrêa2003workconjugacy}.
Table~\ref{tab:helical-perturb02} lists the tip displacements for the solution. 
%
\input{Tables/CantileverPerturbed02}

\hypertarget{sec:helix}{%
\subsubsection{Oscillating Spiral}\label{sec:helix}}
%
To demonstrate the behavior of the
proposed formulations under large rotations, the reference moment value $M$ in Equation~(\ref{eq:fref}) is now increased to \(\lambda=10\) with a large out-of-plane force of \(F=5 \lambda\).
The model discretization uses 100 linear finite elements, and the loading is applied in 200 steps under load factor control.
Figure~\ref{fig:helix} shows the final deformed shape alongside a plot of the tip displacement in the direction of the concentrated force.
These results are in perfect agreement with %values in 
the literature \citep{zupan2003finiteelement,makinen2007total,ghosh2009frameinvariant,lolić2020consistent,harsch2023total}.
