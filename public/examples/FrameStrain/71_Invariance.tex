
\hypertarget{example-1-curvature-invariance}{%
\subsection{Curvature Invariance Under Imposed Rotations}\label{example-1-curvature-invariance}}

This problem highlights the strain invariance %that is
exhibited by the external application of the \texttt{SFIN} transformation to the wrapped
elements. 
Originally presented by \cite{crisfield1999objectivity} for a 2-node element,
the analysis involves evaluating the curvature
\(\boldsymbol{\kappa} = \tau \mathbf{D} + \kappa_\alpha \mathbf{D}_\alpha\)
in an element of length \(L = 1\) under
the following imposed rotations at the first and the last node, respectively:
\[
\boldsymbol{\Lambda}_1= \operatorname{Exp} \left(\begin{array}{c}
1 \\
-0.5 \\
\hphantom{-}0.25
\end{array}\right),
%
\quad \boldsymbol{\Lambda}_n= \operatorname{Exp}\left(\begin{array}{c}
-0.4 \\
\hphantom{-}0.7 \\
\hphantom{-}0.1
\end{array}\right).
\]
The current study extends the investigation by  
%In the test by 
\cite{jelenić1999geometrically}
to \(n\)-node elements %, only two-node elements were used.
with the interior nodes of higher-order elements subjected to 
identity rotations so that \(\boldsymbol{\Lambda}_a = \boldsymbol{1}\) when \(a \notin \{1,n\}\).

New curvatures
\(\boldsymbol{\kappa}^+ = \tau^+ \mathbf{D} + \kappa_\alpha^+\mathbf{D}_\alpha\) are then determined 
in a coordinate system that results from a uniform rotation:
\[
\quad \boldsymbol{R} = \operatorname{Exp}\left(\begin{array}{c}
\hphantom{-}0.2 \\
\hphantom{-}1.2 \\
-0.5
\end{array}\right)
\]
so that the imposed rotations are given by \(\boldsymbol{\Lambda}_a^+ = \boldsymbol{R}\boldsymbol{\Lambda}_a\) at each node.

Table \ref{tab:objective} reports the components of \(\boldsymbol{\kappa}\) and \(\boldsymbol{\kappa}^+\) at the first integration
point.
Because the loading consists only of nodal rotations
imposed in a single increment 
and only strains are evaluated, the results for all three wrapped elements
are identical and reported only once in the table.
The results for two-node elements under the \texttt{None} and \texttt{SFIN} 
transformations match the values %derived
in \cite{crisfield1999objectivity} exactly.
The agreement between the components of \(\boldsymbol{\kappa}\) and \(\boldsymbol{\kappa}^+\)
% \FCF{unclear what the following means} 
for cases using the \texttt{SFIN} isometry
confirms that strain objectivity is restored for the wrapped elements.

\input{Tables/JelenicInvariance.tex}
