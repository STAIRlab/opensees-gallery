\subsection{Curved Cantilever Beam Under Point Force}\label{sec:bathe}
%
The curved cantilever in Figure~\ref{fig:bathe} was first studied
by \cite{bathe1979large}, and has become a staple in the literature
on geometrically nonlinear rods. 
This example is selected to demonstrate the path-independence of the \texttt{Init} interpolation.
The undeformed centerline of the cantilever follows a \(45^\circ\) arc with radius \(R\) given by:
\[
\boldsymbol{x}_0(\xi) = R \sin \xi \frac{\pi}{4L}\, \mathbf{E}_1 
                      + R \left(1 - \cos \xi \frac{\pi}{4L}\right)\, \mathbf{E}_3.
\]
A point load \(\boldsymbol{F} = 600 \, \mathbf{E}_2\) is applied
at the tip, i.e.\ at \(\xi = R\).
There is no closed-form solution to the problem, and it is customary
to present the final displacements at the tip:
\[
\Delta x_i \triangleq \mathbf{E}_i \cdot \left(\boldsymbol{x}(L) - \boldsymbol{x}_0(L)\right).
\]
%
The following parameters are used for the simulations:
\[
\begin{array}{lr}
% Hockling
    R  =& 100 \\ %   ,& A  &= 10 \\
    E  =& 1000 \\ %   ,& I  &= 0.0833 \\
    G  =& 500 \\ %   ,& J  &= 2.16 \\
\end{array}
\qquad\qquad
\begin{array}{ll}
% Hockling
    A  =& 10^4    \\
    I  =& 10^4/12 \\
    J  =& 10^4/6  \\
\end{array}
\]
The analysis uses a discretization with 8 linear (2-node) elements.
The analysis is performed twice for each formulation under consideration,
first with 8 equal load increments and then with 10.
The results are presented in Table~\ref{tab:bathe}.
Only the formulations with the \texttt{Init} interpolation
produce the same tip displacement in both load cases, indicating
an artificial path dependence for all other variants.
% 

\begin{figure}
    \centering
    \includegraphics[width=0.6\textwidth]{Figures/Figure_3}
    \caption{Deformed shape of curved cantilever by \cite{bathe1979large}.}
    \label{fig:bathe}
\end{figure}
%

\input{Tables/CantileverCurved.tex}

% \begin{center}
%\begin{table*}[!h]%
%\caption{}
%\label{tab:bathe}
% \begin{tabular}{llll}
% \midrule
% \cite{bathe1979large}                  & -23.5   & 53.4   & 13.4   \\
% \cite{simo1986threedimensional}        & -23.48  & 53.37  & 13.5   \\
% \cite{cardona1988beam}                 & -23.67  & 53.50  & 13.73  \\
% \cite{crisfield1990consistenta}        & -23.87  & 53.71  & 13.63  \\
% \cite{ibrahimbegović1995finite}        & -23.746 & 53.407 & 13.601 \\
% \cite{ibrahimbegović1995computational} & -23.697 & 53.498 & 13.668 \\
% \cite{schulz2001nonlinear}             &         & 53.56  &        \\ % 4-node element
% \cite{makinen2007total}                & -23.696 & 53.497 & 13.668 \\
% \bottomrule
% \end{tabular}
%\end{table*}
% \end{center}
